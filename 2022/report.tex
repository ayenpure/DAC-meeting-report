\documentclass{article}
\usepackage[utf8]{inputenc}
\usepackage[margin=2cm]{geometry}
\usepackage{fullpage,enumitem,amssymb,amsmath,xcolor,cancel,gensymb,hyperref,graphicx}
\usepackage{indentfirst}

\newcommand{\boit}[1]{\textbf{\textit{#1}}}

\setlength{\parskip}{1em}

\title{DAC Meeting Report}
\author{Abhishek Yenpure}
\date{March $8^{th}$, 2022}

\begin{document}

\maketitle

\section{Research Statement}

My research is at the intersection of scientific visualization and high performance computing.
%
Specifically I focus on the performance of a class of flow visualization algorithms that use ``particle advection.''
%
Particle advection entails displacing particles through a vector field.
%
This vector field could represent a velocity field or some field (e.g. electormagnetic)
that enables the calculation of the velocity of the particles.
%
The particles are then displaced using this velocity.
%
Advecting (displacing) large number of particles efficiently using large supercomputers and novel
accelerator hardware is an active area of research and also constitutes a bulk of my resarch.

\section{Completed Projects}

This section describes the research projects that were completed since the last DAC meeting.

\subsection{A Guide to Particle Advection Performance (Area Exam)}
\noindent \boit{Role:} Primary Author

\noindent \boit{Research Objectives:}
\\
\noindent
(a) To understand the performance of the various components of particle advection.
\\
(b) A survey of various works that optimize particle advection performance.
\\
(c) To formulate a cost function for each of these components.
\\
(d) Reason about performance of flow visualization algorithms.

\noindent \boit{Specific Effort:}
\\
\noindent
(a) Performing the survey of parallel particle advection works.

\noindent \boit{Time:} In submission (8 hrs/week)

\noindent \boit{Counts Towards Thesis:} Yes

\subsection{Particle Advection Speedups from GPU and CPU Parallelism}

\noindent \boit{Role:} Primary Author

\noindent \boit{Research Objectives:}
\\
\noindent
(a) To understand the impact of newer accelerator hardware on the performance of particle advection.
\\
(b) To understand the situations where GPUs perform poor compared to multicore CPUs and developing recommendations
for using accelerators based on given workload.
\\
(c) To quantify the range of performance benefits while using accelerators (GPUs and multicore CPUs).

%\noindent 
\noindent \boit{Specific Effort:}

\noindent (a) Performing the survey of parallel particle advection works.

\noindent \boit{Time:} In Revision (8 hrs/week)

\noindent \boit{Counts Towards Thesis:} Yes

\section{Current Projects}


\subsection{Considering Particle Advection Performance from the Perspective of the Roofline Model}
\noindent \boit{Role:} Primary Author
\noindent \boit{Research Objectives:}

(a) Study the performance of particle advection using the Roofline model

(b) Understanding the impact of various components on the performance

(c) Evaluate different methods to improve the performance ---
like space filling curves for data layout (better memory accesses)
\noindent \boit{Specific Effort:}

(a) Coding : framework for executing experiments

(b) Performance Experiments : collecting perf. stats
\noindent \boit{Time:}
Ongoing (2 hrs/week)
\noindent \boit{Counts Towards Thesis:} Yes

\subsection{Efficient Parallel Particle Advection Via Targeting Devices}
\noindent \boit{Role:} Primary Author
\noindent \boit{Research Objectives:}

(a) Using parallelism (GPU/CPU) to perform particle advection faster

(b) Understand when the GPU performs better compared to the CPU

(c) Formulate a function (Oracle) which makes decisions on where to execute
\noindent \boit{Specific Effort:}

(a) Coding : framework for executing experiments

(b) Performance Experiments : collecting perf. stats
\noindent \boit{Time:}
Ongoing (2 hrs/week)
\noindent \boit{Counts Towards Thesis:} Yes

\subsection{General Particle Advection System}
\noindent \boit{Role:} Primary Author
\noindent \boit{Research Objectives:}

(a) Developing an exascale ready general yet performant particle advection system

(b) Provide great parallelization -- portable shared memory parallelization using VTK-m,
rich distributed memory parallelization(over particles, over data, hybrid)

(c)Demonstrate capability for diverse analysis -- connect with ECP app teams for their flow vis needs
\noindent \boit{Specific Effort:}

(a) Coding : VTK-m + framework for executing experiments

(b) Evaluation : Performance studies

(c) Evaluation : ECP use cases
\noindent \boit{Time:}
Ongoing (10 hrs/week)
\noindent \boit{Counts Towards Thesis:} Yes

\section{Objectives for the Next Year}

Apart from performing revisions for works in submission and getting them accepted,
the following projects need to be completed and submitted. 

\subsection{Considering Particle Advection Performance from the Perspective of the Roofline Model}
(a) Design and execute performance experiments

(b) Collect relevant statistics --- w/ help from B. Norris (UO), C. Yang (LBNL)

(c) Submit paper to $2^{nd}$ tear conference

\subsection{Efficient Parallel Particle Advection Via Targeting Devices}
(a) Design and execute performance experiments

(b) Develop a heursitic based apporach for device selection

(c) Submit paper to $2^{nd}$ tear conference


\subsection{General Particle Advection System}
(a) Complete implementing the system in VTK-m

(b) Begin running performance studies

(c) Connect w/ more ECP applications

(d) Submit paper to Transaction on Visualization and Computer Graphics (TVCG)
\end{document}
