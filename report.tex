\documentclass[12pt,letter]{article}
\usepackage{geometry}\geometry{top=0.75in}
\usepackage{amsmath}
\usepackage{amssymb}
\usepackage{mathtools}
\usepackage{xcolor} % Color words
\usepackage{cancel} % Crossing parts of equations out
\usepackage{tikz}       % Drawing 
\usepackage{pgfplots}   % Other plotting
\usepgfplotslibrary{colormaps,fillbetween}
\usepackage{placeins}   % Float barrier
\usepackage{hyperref}   % Links
\usepackage{tikz-qtree} % Trees
\usepackage{graphicx}
\usepackage{subcaption}
\usepackage{multicol}
\usepackage{graphicx}   % For graphics
\usepackage{parcolumns}
\usepackage{listings}   % lstlisting
\usepackage{pdfpages}
\usepackage{parskip}
\usepackage{bibentry}

\begin{document}
\title{Report for the Dissertation Advisory Committee Meeting}
\author{\parbox{\linewidth}{\centering
	Abhishek Yenpure\\
  Committee : Hank Childs (Advisor), Boyana Norris, Jeewan Choi
	University of Oregon}}
\maketitle
\parskip 0.0625in

\section{Current Research}
In section I will first define the key concepts and terms that are commonly used
in my research. Later, I'll describe my current research in more depth.

\noindent \textbf{Flow Visualization:}
Flow visualization is a prominent branch of scientific visualization that enables
visualization and analysis of fluid dynamics phenomena.
%
This usually involves a computational fluid dynamics simulation which outputs the
data into a mesh and associated fields.
%
Flow visualization is employed the study of aerodynamics, climate modeling, combustion, etc.
%
Domain scientist use the visualization to capture and highlight regions of interest to
better understand the fluid behavior.
%
Most common technique to perform flow visualization is by using \textit{particle advection}.

\noindent \textbf{Particle Advection:}
%
Particle advection is the process of moving massless particles through a flow field,
where the velocity of the particle is calculated using the fields on the problem mesh.
%
This process can be mathematically described by the following equation:
\begin{equation}
\label{eq:euler}
P_{i+1} = P_{i} + h \times v(t_{i},\ P_{i})
\end{equation}
%
Equation \ref{eq:euler} mathematically defines how this process is performed.
%
In the equation, $P_{i}$ is the instantaneous position for the particle at time $t_{i}$
and $P_{i+1}$ is the next position of the particle.
%
The term $v(t_{i},\ P_{i})$ represents the velocity at position $P_{i}$.
%
Finally, $h$ represents the duration of the displacement with the evaluated velocity information.
%
Particle advection is the basis over which various flow visualization techniques are built.

Computationally modeling Equation \ref{eq:euler} has many implicit components  associated with it.
%
For each step of a particle,
we need to be able to \textit{locate} the containing cell for the particle in the problem mesh.
%
After locating the correct cell, we need to be able to \textit{interpolate} the fields on the problem mesh
to calculate the velocity of the particle.
%
Finally, we need to use Equation \ref{eq:euler} to \textit{integrate} the trajectory of the particle
over a series of steps.
%
Equation \ref{eq:euler} is also known as the Euler integration scheme,
which is a first order scheme and requires only one locate and interpolate operation.
%
Higher order integration scheme require more locate and interpolate operations,
which can be determined my the order of the integrator,
but in exchange offer better accuracy.
%
This process needs to be repeated for all particles over the number of steps each
particle needs to take before termination.
%
Hence, it is apparent that particle advection can be a very computationally expensive operation.
%
This fact is compounded by some other explicit factors like --- volume of data,
number of particles required for analysis, initial distribution of particles,
and nature of the flow field.  

My current research focuses on better understanding of all these factors that affect
the performance of particle advection.
%
Currently I am working on my area exam which comprehensively surveys the particle
advection related works that improve or study the  performance of particle advection
while addressing or or more of the above implicit or explicit factors. 
%
I am also studying the shared memory performance of particle advection through
the use of the \textit{roofline model}.
%
This study involves understanding the direct impact of various parameters on the
performance of the algorithm.

Finally, my objective of studying the performance characteristics of particle advection
is to develop a exascale ready flow visualization system.
%
Such a system needs to be highly configurable, extensible and efficient.
%
It needs to be able to handle the analysis requirements from various domains.
%
This system will become a major part of my dissertation, and I plan to demonstrate
its usefulness for the flow analysis needs of the ECP community.

Apart from this my research also involves contributing code to the VTK-m library,
but for the sake of brevity I have left that out from this report.

\nobibliography{pubs}
%\bibliographystyle{unsrt}
\bibliographystyle{ieeetr}

\section{Publications}
\begin{enumerate}
 \item \bibentry{yenpure2019efficient}
 \item \bibentry{pugmire2018perf}
\end{enumerate}

\section{Potential Future Publications}
\begin{enumerate}
  \item Boundary Termination Optimization for Lagrangian Basis Flows
  \\w/ Sudhanshu Sane
  \item Distributed Memory Particle Advection Bake-off
  \\w/ Roba Binyahib
  \item Particle Advection with Device Targeting for Better Performance
  \\w/ Kristi Belcher
  \item Analysis of Particle Advection Performance using the Roofline Model
\end{enumerate}

\section{Activities}

\subsection*{Internship}
\noindent \textbf{Sandia National Laboratories:} \textit{Summer 2019}
\\
This internship involved a self-defined project where I worked towards building
the foundations of my flow visualization system.
%
During this period I also contributed the Finite Time Lyapunov Exponent (FTLE) calculation
filter to VTK-m. 
%
The filter helped Sudhanshu Sane in his efforts to perform a qualitative comparison
of Lagrangian Basis Flows generated to completion to those terminated on boundaries.

\subsection*{Conferences}
\noindent \textbf{EuroVis / EGPGV 2019:} \textit{Porto, Portugal}
\\
Presented my work on efficient point merging, which was also my DRP project. 

\noindent \textbf{VTK-m Code Sprint 2019:} \textit{Albuquerque, NM}
\\
Contributed more code to VTK-m.

\noindent \textbf{Supercomputing 2019:} \textit{Denver, CO}
\\
Volunteered as a Student.

\noindent \textbf{ECP Annual Meeting 2019:} \textit{Houston, TX}
\\
Attended various session on performance profiling successes for ECP applications.

\subsection*{Service}
\noindent \textbf{Early Carrier Program Committee Member}
\\
9th IEEE Large Scale Data Analysis and Visualization (LDAV) symposium, 2019

\noindent \textbf{Technical Paper Review}
\\
In Situ Visualization for Computational Science (textbook by Springer) 2020

\end{document}
